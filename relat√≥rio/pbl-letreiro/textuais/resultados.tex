\chapter{RESULTADOS}
O protótipo passou por alguns problemas de implementação que foram corrigidos depois de algumas discussões e sessões. Entretanto, por fim, após realizar as simulações e todos os testes possíveis depois da implementação, foi considerado o preenchimento de todos os requisítos solicitados. O protótipo cumpre com maestria o deslocamento dos bits utilizando registradores de deslocamento e a manipulação e desenvolvimento de circuitos sequenciais, além de lidar com o erro nas variadas situações possíveis. 

Por outro lado, a manipulação dos códigos de descrição de hardware no mais baixo nível foi bem desenvolvido, relacionando-o com as variadas funções lógicas de saídas. Ademais, foi desenvolvido a habilidade de interpretação, manipulação de códigos númericos e de reconhecimento de padrões e relações entre portas lógicas, desenvolvendo demultiplexadores e multiplexadores.

De forma adicional, mas fora do que foi solicitado, o projeto poderia passar por atualizações para importação de um módulo que selecionasse qualquer palavra informada pelo usuário, não somente a palavra "UEFS".