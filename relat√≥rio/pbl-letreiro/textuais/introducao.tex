\chapter{INTRODUÇÃO}

Embora a utilização de imagens digitais em computador tenha sido datada primeiramente em 1957, por Russel Kirsch \cite{historiaPixel}, a utilização de conjuntos de LEDs para formação de símbolos já era utilizada comumente. A visualização de imagens digitais é uma área da informática que tem como objetivo compreender como as imagens são representadas e processadas por meio de sistemas computacionais. Nesse sentido, é importante compreender como as imagens digitais são formadas a partir de pixels. Por outro lado, indo a baixo nível, utilizando LEDs como representações de pixels, a lógica funciona semelhantemente.

Segundo Gonzalez e Woods (2010) \cite{processamentoDigitalImagem}, uma imagem digital é composta por uma matriz de pixels, sendo cada pixel uma unidade básica de informação da imagem. Cada pixel pode ser representado por um valor numérico que indica sua intensidade de cor ou de cinza. Em uma matriz de LEDs, por exemplo, descrevendo quais LEDs devem ser acionados, uma imagem também é formada. Quanto mais LEDs são utilizados, ou melhor, quanto mais pixels são utilizados, maior a qualidade dos traços, que representa uma melhor qualidade da imagem.

Para visualizar imagens digitais, é necessário o uso de dispositivos eletrônicos que sejam capazes de exibir as informações dos pixels, como telas de computador, smartphones, tablets, entre outros. Esses dispositivos são compostos por pequenos pontos luminosos, que são capazes de exibir diferentes cores e intensidades de luz para formar a imagem. 

De outra forma, um painel eletrônico digital é um dispositivo que exibe informações em tempo real, geralmente de forma visual, em formato de texto, imagens ou gráficos. Esses painéis são usados em diversas áreas, como em eventos esportivos, estações de transporte público, aeroportos, hospitais, empresas, entre outros. A importância dos painéis eletrônicos digitais está relacionada à sua capacidade de fornecer informações precisas e atualizadas em tempo real. Eles permitem a comunicação rápida e eficaz com o público-alvo, fornecendo informações relevantes e importantes de forma clara e acessível.

Além disso, os painéis eletrônicos digitais podem ser personalizados e configurados de acordo com as necessidades específicas de cada ambiente e objetivo, o que os torna uma ferramenta versátil e eficaz para a comunicação e transmissão de informações. Um protótipo de um painel eletrônico digital pode ser desenvolvido com base em diferentes tecnologias, como LED, LCD ou OLED, e pode incluir recursos como touchscreen, Wi-Fi, Bluetooth, entre outros. O objetivo é criar um dispositivo que seja fácil de usar, confiável e eficiente, que possa ser usado em diferentes contextos e para diferentes propósitos.

Portanto, na tentativa de desenvolver competências e habilidades voltadas para a interpretação e utilização de matriz de LEDs, deslocamento de bits, circuitos sequenciais, entre outros, a disciplina Módulo Integrador de Circuitos Digitais da UEFS usa, como problema incentivador, a criação de um protótipo de um painel eletrônico digital.

O problema solicita que o sistema deva receber duas entradas de bits, possibilitando quatro estados diferentes no protótipo. Caso as entradas (Ch0 e Ch1) estejam em nível lógico baixo (0), não deve haver exibição na matriz de LED. Caso as entradas estejam em nível lógico baixo (0) e alto (1), respectivamente, a palavra "UEFS", formada pelos LEDs da matriz, deve ser deslocada da direita para a esquerda. Caso contrário, ou seja, as entradas sejam configuradas em nível lógico alto (1) e baixo (0), a palavra "UEFS" deve ser deslocada da esquerda para a direita. Por fim, caso as entradas estejam ambas acionadas em nível lógico alto (1), a imagem deve ser "pausada" no leitor. As interpretações, escolhas e determinações do projeto foram feitas em sessões tutoriais, utilizando a metodologia Problem-based Learning (PBL).
